\documentclass[usenames,dvipsnames]{beamer}

\input{preamble.tex}

\subtitle{Introduction to Figures and Tables}

\begin{document}
%%%%%%%%%%%%%%%%%%%%%%%%%%%%%%%%%%%%%%%%%%%%%%%%%%%%%%%%%%%%%%%%%%%%%%%%%%%%%%%
%%%%%%%%%%%%%%%%%%%%%%%%%%%%%%%%%%%%%%%%%%%%%%%%%%%%%%%%%%%%%%%%%%%%%%%%%%%%%%%
\begin{frame}
\titlepage
\end{frame}
%%%%%%%%%%%%%%%%%%%%%%%%%%%%%%%%%%%%%%%%%%%%%%%%%%%%%%%%%%%%%%%%%%%%%%%%%%%%%%%
%%%%%%%%%%%%%%%%%%%%%%%%%%%%%%%%%%%%%%%%%%%%%%%%%%%%%%%%%%%%%%%%%%%%%%%%%%%%%%%
\section{Introduction}
\begin{frame}{Outline}
\begin{multicols}{2}
\tableofcontents[currentsection]
\end{multicols}
\end{frame}
%%%%%%%%%%%%%%%%%%%%%%%%%%%%%%%%%%%%%%%%%%%%%%%%%%%%%%%%%%%%%%%%%%%%%%%%%%%%%%%
%%%%%%%%%%%%%%%%%%%%%%%%%%%%%%%%%%%%%%%%%%%%%%%%%%%%%%%%%%%%%%%%%%%%%%%%%%%%%%%
\begin{frame}[fragile]{Starting Point and Goals}
\begin{itemize}
\item You already know some \LaTeX{} and have been introduced to \wllogo
\item Now, we'll learn about positioning {\bf figures} and creating {\bf tables} in \LaTeX{}

\item You'll have a chance to try this in your own document within Overleaf 
\end{itemize}
\vskip 2em
\begin{center}
\fbox{\href{\wlnewdoc{ExampleFigTab.tex}}{%
Click here to open the example document in \wllogo{}}}
\\[1ex]\scriptsize{}
For best results, please use \href{http://www.google.com/chrome}{Google Chrome} or a recent \href{http://www.mozilla.org/en-US/firefox/new/}{FireFox}.
\end{center}
\vskip 2ex
\begin{itemize}
\item Let's get started!
\end{itemize}
\end{frame}
%%%%%%%%%%%%%%%%%%%%%%%%%%%%%%%%%%%%%%%%%%%%%%%%%%%%%%%%%%%%%%%%%%%%%%%%%%%%%%%
%%%%%%%%%%%%%%%%%%%%%%%%%%%%%%%%%%%%%%%%%%%%%%%%%%%%%%%%%%%%%%%%%%%%%%%%%%%%%%%
\section{Figures}
\begin{frame}{Outline}
\begin{multicols}{2}
\tableofcontents[currentsection]
\end{multicols}
\end{frame}
%%%%%%%%%%%%%%%%%%%%%%%%%%%%%%%%%%%%%%%%%%%%%%%%%%%%%%%%%%%%%%%%%%%%%%%%%%%%%%%
%%%%%%%%%%%%%%%%%%%%%%%%%%%%%%%%%%%%%%%%%%%%%%%%%%%%%%%%%%%%%%%%%%%%%%%%%%%%%%%
\subsection{Graphics}
\begin{frame}[fragile]{\insertsubsection}
\begin{itemize}
\item To handle e{\bf x}ternal images, \LaTeX{} requires the \bftt{graphicx} package, which provides the \cmdbs{includegraphics} command. 
\item Just add a \cmdbs{usepackage\{graphicx\}} in the preamble.
\item Supported graphics formats include JPEG, PNG and PDF.
\item \LaTeX{} treats graphics like a chunk of text (a box of a certain width and height).
\end{itemize}
\begin{exampletwouptiny}
\centering
\includegraphics[
  width=0.5\textwidth]{gerbil}

\includegraphics[
  width=0.3\textwidth,
  angle=270]{gerbil}
\end{exampletwouptiny}

\begin{itemize}

\begin{center}
\fbox{\href{\fileuri/gerbil.jpg?dl=1}{Click to download example image}}
\end{center}

\item Upload this image to Overleaf (use the project menu).

\end{itemize}

\tiny{Image license: \href{https://pixabay.com/en/animal-apple-attractive-beautiful-1239390/}{CC0}}
\end{frame}
%%%%%%%%%%%%%%%%%%%%%%%%%%%%%%%%%%%%%%%%%%%%%%%%%%%%%%%%%%%%%%%%%%%%%%%%%%%%%%%
%%%%%%%%%%%%%%%%%%%%%%%%%%%%%%%%%%%%%%%%%%%%%%%%%%%%%%%%%%%%%%%%%%%%%%%%%%%%%%%
\subsection{Simple Example}
\begin{frame}[fragile]{\insertsubsection}

\cmdbs{begin\{figure\}}\\
\cmdbs{centering}\\
\cmdbs{includegraphics[width=1.5in]\{atom.png\}}\\
\cmdbs{end\{figure\}}\\


\begin{figure}
\centering
\includegraphics[width=1.5in]{atom.png}
\end{figure}

\begin{center}
\fbox{\href{\fileuri/lithium.png?dl=1}{Click to download example image}}
\end{center}

\end{frame}
%%%%%%%%%%%%%%%%%%%%%%%%%%%%%%%%%%%%%%%%%%%%%%%%%%%%%%%%%%%%%%%%%%%%%%%%%%%%%%%
%%%%%%%%%%%%%%%%%%%%%%%%%%%%%%%%%%%%%%%%%%%%%%%%%%%%%%%%%%%%%%%%%%%%%%%%%%%%%%%
\begin{frame}[fragile]{Interlude: Optional Arguments}
\begin{itemize}
\item We use square brackets \keystrokebftt{[} \keystrokebftt{]} for optional
arguments, instead of braces \keystrokebftt{\{} \keystrokebftt{\}}.
\item \cmdbs{includegraphics} accepts optional arguments that allow you to transform the
image when it is included. For example, [\bftt{width=0.3\cmdbs{textwidth}}] makes
the image take up 30\% of the width of the surrounding text. You can also use "real" units: [\bftt{width=1.5in}].
\item Other arguments include [\bftt{height}] and [\bftt{angle}].
\item Where do you find out about these? See the Online Resources at the end of this
presentation for links to more information. 
\end{itemize}
\end{frame}
%%%%%%%%%%%%%%%%%%%%%%%%%%%%%%%%%%%%%%%%%%%%%%%%%%%%%%%%%%%%%%%%%%%%%%%%%%%%%%%
%%%%%%%%%%%%%%%%%%%%%%%%%%%%%%%%%%%%%%%%%%%%%%%%%%%%%%%%%%%%%%%%%%%%%%%%%%%%%%%
\subsection{Simple Example with Caption}
\begin{frame}[fragile]{\insertsubsection}

\cmdbs{begin\{figure\}}\\
\cmdbs{centering}\\
\cmdbs{includegraphics[scale=0.5]\{atom.png\}}\\
\cmdbs{caption\{This is Lithium (${}^{6}$Li)\}}\\
\cmdbs{end\{figure\}}\\


\begin{figure}
\centering
\includegraphics[scale=0.5]{atom.png}
\caption{This is Lithium (${}^{6}$Li)}
\end{figure}

\end{frame}
%%%%%%%%%%%%%%%%%%%%%%%%%%%%%%%%%%%%%%%%%%%%%%%%%%%%%%%%%%%%%%%%%%%%%%%%%%%%%%%
%%%%%%%%%%%%%%%%%%%%%%%%%%%%%%%%%%%%%%%%%%%%%%%%%%%%%%%%%%%%%%%%%%%%%%%%%%%%%%%
\subsection[fragile]{Labeling Figures}
\begin{frame}{\insertsubsection}
\begin{itemize}
\item By default, \LaTeX{} will decide where the figure will go within the document (figures and tables ``float''). A Float is an object (typically a table or figure) which cannot be broken over a page.
\item You can caption a float. 
\item With a caption, you can also reference a float using a \cmdbs{ref} and \cmdbs{label} pair:
\end{itemize}
\begin{minipage}{0.45\linewidth}
\inputminted[fontsize=\tiny,frame=single,resetmargins]{latex}{media-graphics.tex}
\end{minipage}
\begin{minipage}{0.5\linewidth}
\includegraphics[width=\textwidth,clip,trim=2in 6.5in 2in 1in]{media-graphics.pdf}
\tiny{Image license: \href{https://pixabay.com/en/animal-apple-attractive-beautiful-1239390/}{CC0}}
\end{minipage}


\end{frame}

%%%%%%%%%%%%%%%%%%%%%%%%%%%%%%%%%%%%%%%%%%%%%%%%%%%%%%%%%%%%%%%%%%%%%%%%%%%%%%%
%%%%%%%%%%%%%%%%%%%%%%%%%%%%%%%%%%%%%%%%%%%%%%%%%%%%%%%%%%%%%%%%%%%%%%%%%%%%%%%
\subsection[fragile]{Placing Figures}
\begin{frame}{\insertsubsection}
%\begin{itemize}

%\item 
\LaTeX{} tries to put a float in the "best" place, to make the document look nice. Overall, it is very good at this. 
\medskip

Because floats are treated as separate entities, and placed on a separate part of the page, away from other text, they tend not to "fit" in the exact place you have placed them in your electronic text. \LaTeX{} is good at finding the optimal location for a float, and placing it there, so you don't have to continue to edit the document, moving figures around whenever you add or remove a bit of text. So we let \LaTeX{} do all the hard work.
%\end{itemize}

\end{frame}
%%%%%%%%%%%%%%%%%%%%%%%%%%%%%%%%%%%%%%%%%%%%%%%%%%%%%%%%%%%%%%%%%%%%%%%%%%%%%%%
%%%%%%%%%%%%%%%%%%%%%%%%%%%%%%%%%%%%%%%%%%%%%%%%%%%%%%%%%%%%%%%%%%%%%%%%%%%%%%%
%\subsection[fragile]{Placing Figures}
\begin{frame}
\begin{itemize}

\item However, sometimes you need to tweak the position of a float. One way is to use the optional position argument in \cmdbs{begin\{figure\}[position]}. \\ For example: \cmdbs{begin\{figure\}[b]}
\end{itemize}
\medskip
\begin{tabular}{ll}

Option & Position \\ \hline
\textbf{h}  & Place the float \textbf{h}ere (more or less) \\
\textbf{t}   & Position at the \textbf{t}op of the page \\
\textbf{b}   & Position at the \textbf{b}ottom of the page \\
\textbf{p}   & Place it on a special \textbf{p}age for floats only \\
\textbf{!}    &Prevent  \LaTeX{} from trying to adjust float location\\
\textbf{H}    & Place the float precisely \textbf{H}ere (like \textbf{h!}) \\
\end{tabular}

\end{frame}
%%%%%%%%%%%%%%%%%%%%%%%%%%%%%%%%%%%%%%%%%%%%%%%%%%%%%%%%%%%%%%%%%%%%%%%%%%%%%%%
%%%%%%%%%%%%%%%%%%%%%%%%%%%%%%%%%%%%%%%%%%%%%%%%%%%%%%%%%%%%%%%%%%%%%%%%%%%%%%%
\section{Tables}


\begin{frame}{Outline}
\begin{multicols}{2}
\tableofcontents[currentsection]
\end{multicols}
\end{frame}
%%%%%%%%%%%%%%%%%%%%%%%%%%%%%%%%%%%%%%%%%%%%%%%%%%%%%%%%%%%%%%%%%%%%%%%%%%%%%%%
%%%%%%%%%%%%%%%%%%%%%%%%%%%%%%%%%%%%%%%%%%%%%%%%%%%%%%%%%%%%%%%%%%%%%%%%%%%%%%%
\subsection{Intro to Tables}
\begin{frame}[fragile]{\insertsubsection}
\begin{itemize}
\item Tables in \LaTeX{} take some getting used to.
\item In general, you should only try to create {\em from scratch} within \LaTeX{} very simple tables. 
\item For real life tables, it will be far easier to export {\em the data} directly from your code, with \LaTeX{} formatting added by your code.
\item Tools exist to help you with this: 
\begin{itemize}
\item \href{https://www.ctan.org/tex-archive/support/excel2latex/}{Excel2\LaTeX{}}
\item \href{https://jeltef.github.io/PyLaTeX/current/}{Py\LaTeX{}} (or DataFrame in Pandas).
\item print(xtable(MyRdata, type = "latex"), file = "MyRtab.tex")
\item \href{https://en.wikibooks.org/wiki/LaTeX/Tables#Using_spreadsheets_and_data_analysis_tools}{Other options}
\end{itemize}
\item Once you have exported the properly formatted data, you can copy and paste it into your document (or just upload it into your project as a separate file).  
\end{itemize}
\end{frame}
%%%%%%%%%%%%%%%%%%%%%%%%%%%%%%%%%%%%%%%%%%%%%%%%%%%%%%%%%%%%%%%%%%%%%%%%%%%%%%%
%%%%%%%%%%%%%%%%%%%%%%%%%%%%%%%%%%%%%%%%%%%%%%%%%%%%%%%%%%%%%%%%%%%%%%%%%%%%%%%
\subsection{A Simple Table}
\begin{frame}[fragile]{\insertsubsection}
\begin{itemize}\item Use the \bftt{tabular} environment from the \bftt{tabularx} package.
\item The argument specifies column alignment --- \textbf{l}eft, \textbf{r}ight, \textbf{r}ight.
\smallskip

\begin{exampletwouptinynoframe}
{\color{OliveGreen}\begin{Verbatim}[fontsize=\scriptsize]
\begin{tabular}{lrr}
Item   & Qty & Unit \$ \\
Widget & 1   & 199.99  \\
Gadget & 2   & 399.99  \\
Cable  & 3   & 19.99   \\
\end{tabular}
\end{Verbatim}}
\end{exampletwouptinynoframe}

\smallskip

Produces the following table:

\begin{exampletwouptinynoframe}
{\color{Blue}\begin{tabular}{lrr}
Item   & Qty & Unit \$ \\
Widget & 1   & 199.99  \\
Gadget & 2   & 399.99  \\
Cable  & 3   & 19.99   \\
\end{tabular}}
\end{exampletwouptinynoframe}
\item Use an ampersand \keystrokebftt{\&} to separate columns and a double backslash \keystrokebftt{\bs}\keystrokebftt{\bs} to start a new row.
\end{itemize}
\end{frame}
%%%%%%%%%%%%%%%%%%%%%%%%%%%%%%%%%%%%%%%%%%%%%%%%%%%%%%%%%%%%%%%%%%%%%%%%%%%%%%%
%%%%%%%%%%%%%%%%%%%%%%%%%%%%%%%%%%%%%%%%%%%%%%%%%%%%%%%%%%%%%%%%%%%%%%%%%%%%%%%
\subsection{Tables: Adding Lines}
\begin{frame}[fragile]{\insertsubsection}
\begin{itemize}

\item You can also specify vertical lines in the optional arguments (use \cmdbs{hline} for horizontal lines).
\smallskip

\begin{exampletwouptinynoframe}
{\color{OliveGreen}\begin{Verbatim}[fontsize=\scriptsize]
\begin{tabular}{|l|r|r|} \hline
Item   & Qty & Unit \$ \\\hline
Widget & 1   & 199.99  \\
Gadget & 2   & 399.99  \\
Cable  & 3   & 19.99   \\\hline
\end{tabular}
\end{Verbatim}}
\end{exampletwouptinynoframe}

\medskip

Produces:
\medskip

\begin{exampletwouptinynoframe}
{\color{Blue}\begin{tabular}{|l|r|r|} \hline
Item   & Qty & Unit \$ \\\hline
Widget & 1   & 199.99  \\
Gadget & 2   & 399.99  \\
Cable  & 3   & 19.99   \\\hline
\end{tabular}}
\end{exampletwouptinynoframe}
\item Note: The spacing and alignment in the typed table does not impact the spacing and alignment in the typeset table.
\end{itemize}
\end{frame}
%%%%%%%%%%%%%%%%%%%%%%%%%%%%%%%%%%%%%%%%%%%%%%%%%%%%%%%%%%%%%%%%%%%%%%%%%%%%%%%
%%%%%%%%%%%%%%%%%%%%%%%%%%%%%%%%%%%%%%%%%%%%%%%%%%%%%%%%%%%%%%%%%%%%%%%%%%%%%%%
\subsection{Tables: Aligning Columns}
\begin{frame}[fragile]{\insertsubsection}
\begin{itemize}

\item In addition to aligning the columns to left, right or center, you can use an @-expression 
\smallskip

\begin{exampletwouptinynoframe}
{\color{OliveGreen}\begin{Verbatim}[fontsize=\scriptsize]
\begin{tabular}{r@{.}l}
  3  & 14159 \\
  16 & 2     \\
  123& 456   \\
\end{tabular}
\end{Verbatim}}
\end{exampletwouptinynoframe}

\medskip

Produces:
\medskip

\begin{exampletwouptinynoframe}
{\color{Blue}\begin{tabular}{r@{.}l}
  3   & 14159 \\
  16  & 2     \\
  123 & 456   \\
\end{tabular}}
\end{exampletwouptinynoframe}
\item What happened? All of the space between the two columns was removed, and a decimal point was inserted in between. A bit odd, but this is a way to make a column of numbers of different precision line up on the decimal point. You do have to separate the numbers (replacing the decimal point with a ampersand).
\end{itemize}
\end{frame}
%%%%%%%%%%%%%%%%%%%%%%%%%%%%%%%%%%%%%%%%%%%%%%%%%%%%%%%%%%%%%%%%%%%%%%%%%%%%%%%
%%%%%%%%%%%%%%%%%%%%%%%%%%%%%%%%%%%%%%%%%%%%%%%%%%%%%%%%%%%%%%%%%%%%%%%%%%%%%%%
\subsection{Tables: \cmdbs{multicolumn} and \cmdbs{multirow}}
\begin{frame}[fragile]{\insertsubsection}
\begin{itemize}

%\item There are additional ways to align columns, and to adjust the spacing between columns.
%\item \cmdbs{multicolumn}

%\smallskip
\begin{exampletwouptinynoframe}
{\color{OliveGreen}\begin{Verbatim}[fontsize=\scriptsize]
\begin{tabular}{ | l | l | r | } \hline
\multicolumn{2}{|c|}{Item} & \multirow{2}{*}{Price (\$)} \\ 
\cline{1-2}
  Animal & Description &  \\ \hline
  Gnat  & per gram & 13.65 \\
        & each     &  0.01 \\
  Gnu   & stuffed  & 92.50 \\
  Emu   & stuffed  & 33.33 \\
  Armadillo & frozen & 8.99 \\ \hline
\end{tabular}
\end{Verbatim}}
\end{exampletwouptinynoframe}

\medskip

Produces:
\bigskip

\begin{exampletwouptinynoframe}
{\color{Blue}\begin{tabular}{ | l | l | r | } \hline
  \multicolumn{2}{|c|}{Item} & \multirow{2}{*}{Price (\$)}\\ \cline{1-2}
  Animal & Description &  \\ \hline
  Gnat  & per gram & 13.65 \\
        & each     &  0.01 \\
  Gnu   & stuffed  & 92.50 \\
  Emu   & stuffed  & 33.33 \\
  Armadillo & frozen & 8.99 \\ \hline
  \end{tabular}}
\end{exampletwouptinynoframe}

\end{itemize}
\end{frame}
%%%%%%%%%%%%%%%%%%%%%%%%%%%%%%%%%%%%%%%%%%%%%%%%%%%%%%%%%%%%%%%%%%%%%%%%%%%%%%%
%%%%%%%%%%%%%%%%%%%%%%%%%%%%%%%%%%%%%%%%%%%%%%%%%%%%%%%%%%%%%%%%%%%%%%%%%%%%%%%
\subsection{Tables: Positioning, Caption and Labeling}
\begin{frame}[fragile]{\insertsubsection}
\begin{itemize}





 \section{Positioning, Captioning, and Labeling a Table}

\begin{table}[b]
\centering
\begin{tabular}{|l|l|r|} \hline
  \multicolumn{2}{|c|}{Item} & \multirow{2}{*}{Price (\$)}\\ \cline{1-2}
  Animal & Description &  \\ \hline
  Gnat  & per gram & 13.65 \\
        & each     &  0.01 \\
  Gnu   & stuffed  & 92.50 \\
  Emu   & stuffed  & 33.33 \\
  Armadillo & frozen & 8.99 \\ \hline
 \end{tabular}
 \caption{\label{tab:prices}Wholesale Prices}
 \end{table}
  


\end{itemize}
\end{frame}
%%%%%%%%%%%%%%%%%%%%%%%%%%%%%%%%%%%%%%%%%%%%%%%%%%%%%%%%%%%%%%%%%%%%%%%%%%%%%%%
%%%%%%%%%%%%%%%%%%%%%%%%%%%%%%%%%%%%%%%%%%%%%%%%%%%%%%%%%%%%%%%%%%%%%%%%%%%%%%%


\section{What's Next?}


\begin{frame}{Outline}
\begin{multicols}{2}
\tableofcontents[currentsection]
\end{multicols}
\end{frame}

%%%%%%%%%%%%%%%%%%%%%%%%%%%%%%%%%%%%%%%%%%%%%%%%%%%%%%%%%%%%%%%%%%%%%%%%%%%%%%%
%%%%%%%%%%%%%%%%%%%%%%%%%%%%%%%%%%%%%%%%%%%%%%%%%%%%%%%%%%%%%%%%%%%%%%%%%%%%%%%
%%%%%%%%%%%%%%%%%%%%%%%%%%%%%%%%%%%%%%%%%%%%%%%%%%%%%%%%%%%%%%%%%%%%%%%%%%%%%%%




\subsection{Resources and Help}
\begin{frame}{\insertsubsection}
\begin{itemize}
\item \href{https://www.overleaf.com/learn}{The Overleaf Learn Wiki} ---
hosts lots of tutorials and reference material
\item \href{http://en.wikibooks.org/wiki/LaTeX}{The \LaTeX{} Wikibook} ---
excellent tutorials and reference material.
\item \href{http://tex.stackexchange.com/}{\TeX{} Stack Exchange} --- ask
questions and get excellent answers incredibly quickly
\item \href{http://www.latex-community.org/}{\LaTeX{} Community} --- a large
online forum
\item \href{http://ctan.org/}{Comprehensive \TeX{} Archive Network (CTAN)} ---
over four thousand packages plus documentation
\item Google will usually get you to one of the above.
\item Ask me! I'm always happy to help with \LaTeX{} questions ricky@virginia.edu
\end{itemize}
\end{frame}

%%%%%%%%%%%%%%%%%%%%%%%%%%%%%%%%%%%%%%%%%%%%%%%%%%%%%%%%%%%%%%%%%%%%%%%%%%%%%%%
%%%%%%%%%%%%%%%%%%%%%%%%%%%%%%%%%%%%%%%%%%%%%%%%%%%%%%%%%%%%%%%%%%%%%%%%%%%%%%%
%%%%%%%%%%%%%%%%%%%%%%%%%%%%%%%%%%%%%%%%%%%%%%%%%%%%%%%%%%%%%%%%%%%%%%%%%%%%%%%


\end{document}

% -- latex understands words, sentences and paragraphs

Words are separated by one or more spaces.  Paragraphs are separated by
one or more blank lines.  The output is not affected by adding extra
spaces or extra blank lines to the input file.

Double quotes are typed like this: ``quoted text''.
Single quotes are typed like this: `single-quoted text'.

Emphasized text is typed like this: \emph{this is emphasized}.
Bold       text is typed like this: \textbf{this is bold}.

-- Adding structure to your document

\section{Hello}

\subsection{World}

\subsection{Foo}

\subsubsection*{Stuff} % star form

\subsubsection*{Results}

-- Labels and cross-references

\label{sec:intro}
\label{sec:method}
\ref{sec:method}

--> maybe introduce the prettyref package here.

-- Mathematics

Inline mathematics: $x + y < 7$.

'Displayed' mathematics:
\begin{equation}
\end{equation}

\begin{equation*}
\end{equation*}

\begin{align}
\end{align}

-- Figures

- Need the graphicx package. (X-ternal graphics)

- here we can start introducing options

\includegraphics[width=\textwidth]{}

- where do you find out about these options? --> link to the Wikibook

-- Floating Figures 

- Floats include Figures and Tables. LaTeX works hard to place floats in a place that will look good, but we can override these choices 

\begin{figure}
\includegraphics{...}
\caption{\label{}Here is a caption.}
\end{figure}

-- Tables

- not the nicest part of LaTeX
- Format the contents of the table outside of LaTeX (excel, python, R, or use an online table generator).

\usepackage{tabularx}

\begin{tabular}{llr}
Item & Quantity & Price (\$) & Amount
Widget & 1 &
\end{tabular}

Bonus points: check out the fp package and the spreadtab package.

-- Document Classes

a .cls file

article

some journal templates come with one

-- Bibliographies



-- For Typesetting Geeks

- dashes: -, --, ---

- ellipsis.

- controlling spaces: ~, \ , \,, \@

- spacing after periods (et al., etc.)

- Nested quotation marks: ``\,`
\vskip 2ex
\item Use the \emph{star form} to display an equation without a number.
\begin{exampletwouptiny}
\begin{equation*}
F(x) = \int_{a}^{x}{f(t) dt}
\end{equation*}
\end{exampletwouptiny}

\begin{itemize}
\item \bftt{equation} and \bftt{equation*} are called \emph{environments}.
\begin{itemize}
  \item The \cmdbs{begin} and \cmdbs{end} commands define the environment.
  \item The \cmd{\$} also starts and ends an environment.
  \item Some commands are defined only within certain environments.
  \item Some commands behave differently in different environments.
\end{itemize}
\end{itemize}
\end{block}
\begin{center}
\fbox{\href{http://ctan.org/}{The Comprehensive \TeX Archive Network (CTAN)}}
\end{center}

%%%%%%%%%%%%%%%%%%%%%%%%%%%%%%%%%%%%%%%%%%%%%%%%%%%%%%%%%%%%%%%%%%%%%%%%%%%%%%%
%%%%%%%%%%%%%%%%%%%%%%%%%%%%%%%%%%%%%%%%%%%%%%%%%%%%%%%%%%%%%%%%%%%%%%%%%%%%%%%
%%%%%%%%%%%%%%%%%%%%%%%%%%%%%%%%%%%%%%%%%%%%%%%%%%%%%%%%%%%%%%%%%%%%%%%%%%%%%%%
\subsection{Typography tweaks}
\begin{frame}{\insertsubsection}
\begin{tabular}{lll}
& character name & used mainly for \ldots \\\hline
\bftt{\bs} & backslash                 & commands, tables \\
\bftt{\{}  & open brace                & commands \\
\bftt{\}}  & close brace               & commands \\
\bftt{\%}  & percent sign              & comments \\
\bftt{\#}  & hash (pound / sharp) sign & custom commands \\
\bftt{\$}  & dollar sign               & equations \\
\bftt{\_}  & underscore                & equations (subscripts) \\
\bftt{\^}  & caret                     & equations (superscripts) \\
\bftt{\&}  & ampersand                 & tables \\
\bftt{\~}  & tilde                     & spacing \\
\end{tabular}
\end{frame}

%\item We've used several environments:
%\vskip 1ex
%{\scriptsize
%\begin{tabular}{ll}
%\cmdbs{begin}\bftt{\{document\}}\ldots\cmdbs{end}\bftt{\{document\}} &
%  document environment \\
%\cmdbs{begin}\bftt{\{itemize\}}\ldots\cmdbs{end}\bftt{\{itemize\}} &
%  itemized list environment \\
%\bftt{\$\ldots\$}     & \emph{in-text} math environment \\
%\bftt{\$\$\ldots\$\$} & \emph{displayed} math environment \\
%\cmdbs{begin}\bftt{\{equation\}}\ldots\cmdbs{end}\bftt{\{equation\}} &
%  displayed math environment w/ number
%\end{tabular}
%}
